\chapter{Solution Design}
\label{chap:SolutionDesign}

\section{Requirements}

\subsection{Functional Requirements}

The solution design focuses on an AI-enabled production order release system with the following key functional requirements:

\begin{itemize}
    \item \textbf{AI-Enabled Case Management}: The system should leverage artificial intelligence to analyze production order descriptions and compare them with historical data to make intelligent release decisions.
    
    \item \textbf{Historical Data Analysis}: The system must fetch and analyze production order information from the past month to identify patterns and similarities in component usage and production requirements.
    
    \item \textbf{Component Availability Checking}: Instead of traditional BOM (Bill of Materials) verification, the system should check past production orders with similar components to assess feasibility.
    
    \item \textbf{ATP (Available-to-Promise) Validation}: The system must perform ATP checks for missing components to ensure production feasibility before order release.
    
    \item \textbf{Intelligent Recommendations}: The system should provide alternative solutions based on past events and historical production data.
    
    \item \textbf{Automated Testing Framework}: Integration with Cucumber framework for comprehensive automated testing of the solution.
\end{itemize}

\subsection{Non-Functional Requirements}

\begin{itemize}
    \item \textbf{Integration Capability}: Seamless integration with SAP S/4HANA Public Cloud Edition
    \item \textbf{Performance}: Real-time processing of production order release decisions
    \item \textbf{Scalability}: Ability to handle multiple production orders simultaneously
    \item \textbf{Reliability}: High availability and fault tolerance
\end{itemize}

\section{System Architecture}

\subsection{Overview}

The solution architecture is designed around a multi-layered approach that integrates AI capabilities with SAP S/4HANA's existing infrastructure. The core components include:

\begin{itemize}
    \item \textbf{AI Agent Layer}: Central intelligence unit that processes production orders and makes release decisions
    \item \textbf{Joule Integration}: Independent copilot functionality that provides AI-powered assistance
    \item \textbf{OData Services Layer}: Interface for accessing historical production data and SAP system information
    \item \textbf{Testing Framework}: Cucumber-based automated testing infrastructure
    \item \textbf{SAP S/4HANA Integration}: Native integration with the existing SAP system for automated production order release
\end{itemize}


\section{Workflow}

The production order release workflow follows a streamlined process that emphasizes AI-driven decision making:

\subsection{Primary Workflow}

\begin{enumerate}
    \item \textbf{Order Reception}: Production order is received by the system
    \item \textbf{AI Analysis}: The AI agent analyzes the order description and compares it with historical data
    \item \textbf{Historical Data Retrieval}: System fetches relevant production order information from the past month
    \item \textbf{Component Analysis}: Instead of BOM checking, the system analyzes past production orders with similar components
    \item \textbf{ATP Validation}: Available-to-Promise check for missing components
    \item \textbf{Recommendation Generation}: AI generates release recommendations based on historical patterns
    \item \textbf{Alternative Suggestions}: System provides alternative solutions based on past events
    \item \textbf{Order Release Decision}: Final decision on whether to release the production order
\end{enumerate}

\subsection{Data Flow}

The data flow emphasizes the integration between different system components:

\begin{itemize}
    \item Production order data flows from SAP S/4HANA to the AI agent
    \item Historical data is retrieved via OData services from HANA database
    \item Joule provides additional AI capabi   lities and tool selection
    \item Recommendations flow back to the SAP system for final processing
\end{itemize}
