\chapter{Evaluation}
\label{chap:Evaluation}

\section{Evaluation Criteria}

The evaluation of the Production Order Release Agent focuses on three key performance indicators that directly address the research questions outlined in \Cref{sec:Introduction}:

\begin{itemize}
    \item \textbf{Accuracy}: The agent's ability to correctly identify missing components and recommend appropriate alternatives
    \item \textbf{Efficiency}: Reduction in manual effort required for production order release decisions
    \item \textbf{Reliability}: Consistency in providing useful recommendations across different scenarios
\end{itemize}

These criteria are measured against the baseline of manual production order release processes currently used in SAP S/4HANA Public Cloud environments.

\section{Test Setup}

\subsection{Test Environment}

The evaluation was conducted using SAP S/4HANA Public Cloud Edition with the following production order management transactions:

\begin{itemize}
    \item \textbf{Create Production Order (CO01)}: Used to generate test production orders with various component configurations
    \item \textbf{Change Production Order (CO02)}: Applied to modify existing orders and simulate different scenarios
    \item \textbf{Display Production Order (CO03)}: Utilized to monitor order status and component availability
\end{itemize}

These transactions provide comprehensive functionality for production order lifecycle management in SAP S/4HANA, as documented in the official SAP documentation \cite{sapco01}.

\subsection{Test Dataset}

A comprehensive test dataset was created to evaluate the agent's performance across different production order scenarios. The dataset includes 14 distinct production orders with varying component configurations:

\begin{table}[h]
\centering
\caption{Test Dataset Overview}
\label{tab:test-dataset}
\begin{tabular}{@{}llllll@{}}
\toprule
\textbf{Order ID} & \textbf{FrameA} & \textbf{Battery} & \textbf{Wheel} & \textbf{Released} & \textbf{Missing Components} \\
\midrule
1025282 & FrameB & BatteryA & - & N & MATS-BATTERY \\
1025283 & FrameA & BatteryA & - & N & MATS-BATTERY \\
1025284 & FrameA & BatteryA & - & N & MATS-BATTERY \\
1025285 & FrameA & BatteryA & WheelA & N & MATS-BATTERY \\
1025286 & FrameA & - & - & Y & none \\
1025281 & FrameA & BatteryA & - & Y & MATS-BATTERY \\
1025287 & FrameA & BatteryA & WheelA & N & MATS-BATTERY \\
1025288 & FrameB & BatteryA & WheelB & N & MATS-BATTERY \\
1025289 & - & BatteryA & WheelB & N & MATS-FRAME \\
1025290 & - & - & WheelC & N & MATS-FRAME, MATS-BATTERY \\
1025291 & FrameA & BatteryB & WheelA & Y & none \\
1025292 & FrameA & BatteryB & WheelB & N & MATS-WHEEL \\
1025293 & FrameA & BatteryB & WheelC & N & MATS-WHEEL \\
\bottomrule
\end{tabular}
\end{table}

\subsection{Test Scenario Explanations}

The dataset was designed to test various scenarios and edge cases:

\begin{table}[h]
\centering
\caption{Test Scenario Descriptions}
\label{tab:test-scenarios}
\begin{tabular}{@{}ll@{}}
\toprule
\textbf{Order ID} & \textbf{Test Purpose} \\
\midrule
1025282 & Test agent with different frame type (FrameB) and missing battery \\
1025283 & Baseline case with standard frame and missing battery \\
1025284 & Duplicate scenario to test consistency in recommendations \\
1025285 & Test agent with multiple components including wheel \\
1025286 & \textbf{Control case}: Complete order with no missing components \\
1025281 & \textbf{Edge case}: Released order despite missing battery (manual override) \\
1025287 & Test agent with complete component set but still missing battery \\
1025288 & Test agent with different component combination (FrameB + WheelB) \\
1025289 & Test agent with missing frame component \\
1025290 & Test agent with multiple missing components (frame and battery) \\
1025291 & \textbf{Control case}: Complete order with alternative battery type \\
1025292 & Test agent with missing wheel component \\
1025293 & Test agent with different wheel type and missing component \\
\bottomrule
\end{tabular}
\end{table}

The dataset was designed to test various scenarios including:
\begin{itemize}
    \item \textbf{Control Cases}: Orders with complete component sets (1025286, 1025291)
    \item \textbf{Missing Component Cases}: Orders with single missing components (1025282-1025285, 1025287-1025288)
    \item \textbf{Multiple Missing Components}: Orders with multiple missing components (1025290)
    \item \textbf{Edge Cases}: Released orders with missing components (1025281)
    \item \textbf{Consistency Testing}: Duplicate scenarios to test recommendation consistency (1025283, 1025284)
    \item \textbf{Variety Testing}: Different component combinations and types
\end{itemize}

\subsection{Testing Methodology}

The evaluation process involved:

\begin{enumerate}
    \item \textbf{Baseline Measurement}: Recording manual processing time and decision accuracy for each test case
    \item \textbf{Agent Testing}: Deploying the Production Order Release Agent to process the same test cases
    \item \textbf{Performance Comparison}: Analyzing differences in processing time, accuracy, and recommendation quality
    \item \textbf{Error Analysis}: Identifying patterns in cases where the agent provided suboptimal recommendations
\end{enumerate}

\section{Results}

\subsection{Overall Performance}

The Production Order Release Agent was tested against the dataset to evaluate its ability to retrieve relevant materials in a more complex dataset across a wider range of time. The evaluation focused on the agent's performance in identifying missing components and recommending appropriate alternatives.

\subsection{Component Recommendation Analysis}

Based on the test results, the agent demonstrated the following performance characteristics:

\begin{itemize}
    \item \textbf{Generally Successful}: The agent was able to retrieve relevant materials in most cases across the test dataset
    \item \textbf{Occasional Redundancy}: In some instances, the agent recommended the same components that were currently missing, indicating areas for improvement in the recommendation logic
    \item \textbf{Complex Dataset Handling}: The agent showed capability in handling the diverse scenarios present in the test dataset
\end{itemize}

\subsection{Key Findings}

The evaluation revealed that:

\begin{enumerate}
    \item The agent successfully identified missing components in the majority of test cases
    \item Alternative component recommendations were provided for most scenarios
    \item Some cases showed redundant recommendations where the agent suggested components that were already identified as missing
    \item The agent demonstrated improved performance compared to manual processes in terms of speed and consistency
\end{enumerate}

\section{Discussion of Results}

\subsection{Strengths of the Agent}

The evaluation results demonstrate several key strengths of the Production Order Release Agent:

\begin{enumerate}
    \item \textbf{Effective Material Retrieval}: The agent successfully retrieved relevant materials in most test cases, addressing the core challenge of incomplete BOM data
    \item \textbf{Consistent Performance}: The agent maintained reliable performance across diverse test scenarios
    \item \textbf{Automated Processing}: The agent reduced manual effort required for production order release decisions
\end{enumerate}

\subsection{Limitations and Areas for Improvement}

The evaluation also revealed areas where the agent's performance could be enhanced:

\begin{enumerate}
    \item \textbf{Recommendation Redundancy}: The agent occasionally recommended components that were already identified as missing, suggesting the need for improved validation logic
    \item \textbf{Recommendation Logic}: There is room for improvement in the logic that determines alternative component suggestions
    \item \textbf{Complex Scenario Handling}: The agent could benefit from enhanced logic for handling cases with multiple missing components
\end{enumerate}

\subsection{Implications for Production Environments}

The results suggest that the Production Order Release Agent can provide value in real-world manufacturing environments:

\begin{itemize}
    \item \textbf{Operational Efficiency}: The agent reduces manual effort in production order release processes
    \item \textbf{Improved Decision Making}: Automated recommendations help production supervisors make informed decisions
    \item \textbf{Scalability}: The agent's performance suggests it can handle larger, more complex production order volumes
\end{itemize}

\subsection{Future Enhancement Opportunities}

Based on the evaluation results, several areas for future development have been identified:

\begin{enumerate}
    \item \textbf{Enhanced Validation Logic}: Implement additional checks to prevent redundant component recommendations
    \item \textbf{Improved Recommendation Algorithms}: Develop more sophisticated logic for alternative component suggestions
    \item \textbf{Multi-Component Scenarios}: Develop specialized logic for handling complex cases with multiple missing components
\end{enumerate}

The evaluation demonstrates that the Production Order Release Agent shows promise in addressing the core challenges identified in the problem statement while providing operational benefits in SAP S/4HANA Public Cloud environments. However, there are opportunities for further refinement to improve recommendation accuracy and reduce redundant suggestions.